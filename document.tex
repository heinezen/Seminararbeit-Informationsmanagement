%!TEX encoding = UTF-8 Unicode
\documentclass[12pt]{scrartcl}
%\usepackage[applemac]{inputenc} % Mac-Umlaute direkt verwenden öäüß
%\usepackage[isolatin]{inputenc} % PC-Umlaute direkt verwenden 
\usepackage[utf8]{inputenc} % Unicode funktioniert unter Windows, Linux und Mac
\usepackage[T1]{fontenc}
\usepackage{times}
\usepackage{ngerman}
\usepackage{url}
\usepackage[scaled]{helvet}
\usepackage{a4wide}
\usepackage{rotating}
\usepackage{listings}\lstset{breaklines=true,breakatwhitespace=true,frame=leftline,captionpos=b,xleftmargin=6ex,tabsize=4,numbers=left,numberstyle=\ttfamily\footnotesize,basicstyle=\ttfamily\footnotesize}
\setlength{\parindent}{0em}
\setlength{\parskip}{1.2ex plus 0.5ex minus 0.5ex}
\pagestyle{plain}

\begin{document}
	
	\newpage
	\thispagestyle{empty}
	\begin{center}\Large
		Universität Hamburg \par
		Institut für Wirtschaftsinformatik
		\vfill
		Seminararbeit
		\vfill
		{\Large\textsf{\textbf{TeX oder Tod}}\par}
		\vfill
		vorgelegt von 
		\par\bigskip
		Christoph Heine \par
		Matrikelnummer 6524129 \par
		Studiengang Wirtschaftsinformatik \par
		Daniel M. \par
		Matrikelnummer 1234567 \par
		Studiengang Wirtschaftsinformatik
		\vfill
		eingereicht am \today
		\vfill 
		Betreuer:  \par
		Erstgutachter:  \par
	\end{center}
	
	\newpage
	\tableofcontents
	
	\newpage
	\section*{Zusammenfassung}
	Zitat\footnote{\cite{IEEE2013}}.
	
	\newpage
	\section{Aller Anfang ist schwer}
	\subsection{Daniels Reise zum Informatikum}
	\subsection{Fred greift ein}
	\subsection{Ein TeX kommt selten mit GUI}
	\subsection{Ohne Voß nichts loß}
	
	\newpage
	\section{Verbündete}
	\subsection{Himbeereis zu Mittag}
	\subsection{Angebot und Annahme}
	\subsection{AFK: Anti-Fred-Koalition}
	\subsection{TeXnische Probleme}
	
	\newpage
	\section{Es wird eng}
	\subsection{Scheitern bei BWL...}
	\subsection{...Sieg bei Informatik}
	\subsection{Freds Untergang}

	\newpage
	\addcontentsline{toc}{section}{Literaturverzeichnis}
	\begin{raggedright}%schaltet Blocksatz ab, erzeugt ein stimmigeres Schriftbild im Literaturverzeichnis
		\begin{thebibliography}{XXXXXXXX}
		\bibitem[IEEE2013]{IEEE2013} Alex Biryukov, Ivan Pustogarov, Ralf-Philipp Weinmann: Trawling for Tor Hidden Services: Detection, Measurement, Deanonymization. IEEE Symposium on Security and Privacy, San Francisco, 2013, S. 80-94.
		\end{thebibliography}
	\end{raggedright}
	
	\newpage
	
	\section{Selbstständigkeitserklärung}
	
	Ich versichere, dass ich die Arbeit selbstständig verfasst und keine anderen als die angegebenen Hilfsmittel -- insbesondere keine im Quellenverzeichnis nicht benannten Internetquellen -- benutzt habe, die Arbeit vorher nicht in einem anderen Prüfungsverfahren eingereicht habe und die eingereichte schriftliche Fassung der auf dem elektronischen Speichermedium entspricht.
	
	Ich bin mit der Einstellung der Arbeit in den Bestand der Bibliothek der WISO-Fakultät einverstanden.
	
	Hamburg, den \today
	
	\bigskip
	(Unterschrift)\\
	Christoph Heine
	
	\bigskip
	\bigskip
	(Unterschrift)\\
	Daniel M.
\end{document}